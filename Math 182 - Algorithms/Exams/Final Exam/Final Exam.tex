\documentclass[11pt]{amsart}

%%%%%%%%%%%%%%%%%%%%%%%%%%%%%%%%%%%%%%%%
% Packages
\usepackage{amsmath}
\usepackage{amsthm}
\usepackage{amssymb}
\usepackage{enumerate}
\usepackage{fullpage}
\usepackage{csquotes}
\usepackage{graphicx}

\usepackage{clrscode3e}

\usepackage{listings}
\usepackage{xcolor}

\definecolor{codegreen}{rgb}{0,0.6,0}
\definecolor{codegray}{rgb}{0.5,0.5,0.5}
\definecolor{codepurple}{rgb}{0.58,0,0.82}
\definecolor{backcolour}{rgb}{0.95,0.95,0.92}

\lstdefinestyle{mystyle}{
    backgroundcolor=\color{backcolour},   
    commentstyle=\color{codegreen},
    keywordstyle=\color{magenta},
    numberstyle=\tiny\color{codegray},
    stringstyle=\color{codepurple},
    basicstyle=\ttfamily\footnotesize,
    breakatwhitespace=false,         
    breaklines=true,                 
    captionpos=b,                    
    keepspaces=true,                 
    numbers=left,                    
    numbersep=5pt,                  
    showspaces=false,                
    showstringspaces=false,
    showtabs=false,                  
    tabsize=2
}

\lstset{style=mystyle}

\theoremstyle{theorem}
\newtheorem{exercise}{Exercise}
\newtheorem{question}{Question}
\newtheorem*{claimunnumbered}{Claim}

%%%%%%%%%%%%%%%%%%%%%%%%%%%%%%%%%%%%%%%%
%Math Macros
\newcommand\N{\mathbb{N}}
\newcommand\Z{\mathbb{Z}}
\newcommand\Q{\mathbb{Q}}
\newcommand\R{\mathbb{R}}
\newcommand\C{\mathbb{C}}
\newcommand\E{\mathbb{E}}
\renewcommand\P{\mathbb{P}}
\newcommand{\floor}[1]{\left\lfloor #1 \right\rfloor}
\newcommand{\ceil}[1]{\left\lceil #1 \right\rceil}
\newcommand\Mod{\operatorname{mod}}

\DeclareMathOperator\Uniform{Uniform}
\DeclareMathOperator\Geometric{Geometric}
\DeclareMathOperator\Normal{Normal}
\DeclareMathOperator\Exponential{Exponential}
\DeclareMathOperator\Erlang{Erlang}
\DeclareMathOperator\Range{Range}
\DeclareMathOperator\Cov{Cov}
\DeclareMathOperator\Var{Var}
\DeclareMathOperator\Inv{Inv}


%%%%%%%%%%%%%%%%%%%%%%%%%%%%%%%%%%%%%%%%
%homework macros
\newcommand\duedate{July 31, 2020} %Change this accordingly
\newcommand\homeworknumber{2} %Change this accordingly

% Tikz for graphs and such
\usepackage{tikz}
\usetikzlibrary{calc}
\usetikzlibrary{graphs,graphs.standard}
\tikzstyle{vertex}=[circle, draw, fill=black, inner sep=0pt, minimum width=3pt]

%\author{Your name here}
%\email{your.email.address.here@ucla.edu}

\title{Math182 Final %\#\homeworknumber
\\ Due \duedate}

%%%%%%%%%%%%%%%%%%%%%%%%%%%%%%%%%%%%%%%%



\begin{document}
\maketitle


\begin{question}
(5pts) Write an algorithm which implements depth-first search without recursion. You might find a \emph{stack} data structure to be useful. Your algorithm should take as input a graph $G=(V,E)$ and upon termination should successfully assign the \emph{predecessor}, \emph{discovery time}, and \emph{finishing time} attributes to all vertices. You can also use the \emph{color} attribute.
\end{question}

\begin{question}
(5pts) Give the optimal parenthesization for a matrix chain product $\langle A_1,A_2,A_3,A_4, A_5, A_6\rangle$, where
\begin{itemize}
\item $A_1$ has size $5\times 10$
\item $A_2$ has size $10\times 3$
\item $A_3$ has size $3\times 12$
\item $A_4$ has size $12\times 5$
\item $A_5$ has size $5\times 50$
\item $A_6$ has size $50\times 6$
\end{itemize}
\end{question}

\begin{question}
(5pts) Suppose $G=(V,E)$ is a connected, undirected graph. Show that if an edge $(u,v)$ is contained in some minimum spanning tree of $G$, then it is a light edge crossing some cut of the graph.
\end{question}

\begin{question}
(5pts) Describe an efficient algorithm that, given a set $\{x_1,x_2,\ldots,x_n\}$ of points on the real line, determines the smallest set of unit-length closed intervals that contains all of the given points. Argue that your algorithm is correct.
\end{question}

\begin{question}
(True/False) For each of the following statements indicate whether they are \textbf{true} or \textbf{false}. Each question is worth 2pts, a blank answer will receive 1pt. Recall that ``true'' means ``always true'' and ``false'' means ``there exists a counterexample''.
\begin{enumerate}
\item Since \proc{Max-Heapify} runs in $O(\lg n)$ time, the process \proc{Build-Max-Heap} runs in $\Theta(n\lg n)$ time.
\item In the rod-cutting problem, the naive recursive solution \proc{Cut-Rod} runs in polynomial time.
\item If a problem has a greedy algorithm solution, then it necessarily does not have a dynamic programming solution; and vice versa.
\item The algorithm \proc{BFS} visits every vertex of the graph.
\item During DFS, if a vertex $v$ has its color changed from white to gray while $u$ is gray, then $\attrib{u}{f}<\attrib{v}{f}$.
\end{enumerate}
\end{question}





\end{document}